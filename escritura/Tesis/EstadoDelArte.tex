% \part{Estado del arte}
\chapter{Segmentación de imágenes medicas}\label{chapter:segmencation}
Con el creciente uso de las imágenes médicas para el diagnóstico, planeamiento de tratamientos y estudios clínicos, se ha vuelto casi obligatoria la utilización de computadoras para asistir en el análisis de dichas imágenes. En este contexto, uno de los objetivos del diagnóstico asistido por computadoras es acelerar la obtención de resultados precisos \citep{Sharma2010}. Una de las etapas más importantes en este proceso es la segmentación ya que las etapas posteriores dependen en gran medida de la precisión de sus resultados. Para la delineación de estructuras anatómicas y otras regiones de interés es necesario contar con algoritmos confiables.
El objetivo de este capítulo es presentar las etapas del procesamiento de imágenes y algunas estrategias de segmentación, así como un relevamiento del estado del arte. En la sección \ref{section:procesamiento} se describen  las etapas del procesamiento de imágenes, enfocadas particularmente en su aplicación a la medicina. Finalmente en la sección \ref{section:algoritmos} se realiza una revisión de los algoritmos de segmentación más comúnmente utilizados sobre imágenes médicas.

\section{Procesamiento de imágenes}\label{section:procesamiento}
Los sistemas de procesamiento de imágenes médicas posibilitan obtener información de interés a partir de estudios de diversa índole, lo que permite a los profesionales médicos mejorar la precisión de sus diagnósticos y planificar tratamientos o cirugías. El principal objetivo de estos sistemas es obtener una imagen de salida que sea una representación precisa de la señal de entrada, para luego analizarla y extraer de ella la mayor cantidad de información de diagnóstico posible \citep{dougherty2009digital}. Sin importar el dominio de análisis, el procesamiento de imágenes se compone en general de 5 etapas que se pueden observar en la figura \ref{fig:etapas_del_procesamiento}.

\begin{figure}[h!]
\centering
\includegraphics[scale=0.3]{images/procesamiento.png}
\caption{Etapas del procesamiento de imágenes}
\label{fig:etapas_del_procesamiento}
\end{figure}

La primera etapa del procesamiento de imágenes es la captura. En el ámbito médico existen diferentes modalidades de captura. Una de estas modalidades es la resonancia magnética (RM). La RM es una de las técnicas más utilizadas para el diagnóstico y seguimiento de distintas enfermedades \citep{prince2006medical}. En la figura \ref{fig:scanner:mri} se puede observar un escáner de resonancia magnética.  En este tipo de modalidad se utilizan campos magnéticos muy potentes para alinear la magnetización nuclear de los átomos  de hidrógeno del agua en el cuerpo. Unos campos de radiofrecuencia (RF) se utilizan luego para alterar sistemáticamente el alineamiento de esa magnetización, provocando que los átomos de hidrógeno generen un campo magnético rotacional detectable por el escáner \citep{novelline2004squire}.

\begin{figure}[h!]
\centering
\includegraphics[scale=0.3]{images/scanner.jpg}
\caption{Scanner de resonancia magnética}
\label{fig:scanner:mri}
\end{figure}

Esta es una técnica flexible y dinámica que permite obtener imágenes de contraste variable utilizando secuencias de pulso y ajustando los parámetros de captura de la imagen correspondientes al tiempo longitudinal (T1) y al tiempo transversal (T2). El contraste de una imagen de resonancia magnética es un factor dependiente del ajuste de la secuencia de pulsos. Las imágenes obtenidas mediante este procedimiento son generalmente más eficientes para detectar anormalidades en el cerebro durante las etapas tempranas de una enfermedad así como también son excelentes para la detección de tumores cerebrales o infecciones. En particular, MR es muy útil para detectar enfermedades en la materia blanca tales como esclerosis múltiple, encefalopatias, y encefalitis post infecciosa. Los determinantes principales del contraste y la intensidad de la señal son los tiempos de T1 y T2. El contraste es diferente entre imágenes T1 y T2 y las lesiones patológicas pueden separarse en grupos de acuerdo a las características de las dos imágenes mencionadas (T1 y T2):

\begin{table}[hp]
	\centering
	\begin{tabular}{c|c c}
	Patología & Contraste en T1 & Contraste en T2   \\ 
	\hline Masa solida & Brillante & Oscura  \\ 
	Grasa & Oscura & Brillante  \\ 
	Quiste &	Brillante &	Oscura  \\ 
	\end{tabular} 
\end{table}


\section{Algoritmos de segmentación}\label{section:algoritmos}
La segmentación de imágenes es un proceso a través del cual se divide una imagen en regiones con propiedades similares [pham2000current]. Dichas propiedades son determinadas por el algoritmo utilizado para tal efecto, y pueden involucrar criterios basados en intensidades de gris, características de textura, color o contraste. El objetivo de la segmentación es simplificar o cambiar la representación de una imagen en información significativa y sencilla de analizar [Barghout and Lee, 2003]. Tradicionalmente, la salida de un algoritmo de segmentación consiste en una máscara booleana que representa con 1s los puntos de la imagen que pertenecen a la región de interés, y con 0s cualquier otro punto. En particular, la segmentación de imágenes médicas se utiliza para estudiar estructuras anatómicas o patológicas tales como tumores, lesiones u otras anormalidades. También se aplica para medir el volumen de una región y estudiar, por ejemplo, el crecimiento o reducción de un tumor, y luego ayudar en la planificación de tratamientos de radioterapia y calcular dosis de radiación [Sharma and Aggarwal, 2010]. 
La segmentación es uno de los problemas más difíciles dentro del procesamiento de imágenes ya que la mayoría de los objetos reales tienen formas, bordes y morfología compleja [Wu and Chung, 2007], ademas de tener que lidiar con los problemas comunes de procesamiento de imagenes como el ruido y los problemas de volumen parcial mencionados en la sección anterior. 
Los algoritmos de segmentación pueden dividirse en dos grandes grupos, los métodos supervisados, que requieren datos de entrenamientos que el algoritmo utiliza como referencia para realizar la segmentación, y los no supervisados, algoritmos automáticos que realizan un agrupamiento sin tener datos previos [pham2000current]
En las siguientes subsecciones se explicarán algunos de los principales algoritmos de segmentación relevados del estado del arte para cada grupo.

\subsection{Umbralado}
Los algoritmos de segmentación por umbralado,  thresholding en inglés, permiten dividir una imagen escalar en dos regiones de interés mediante la separación de las intensidades [Weszka, 1978]. Este tipo de enfoques busca determinar un valor de intensidad, denominado umbral, que es luego utilizado para separar las diferentes regiones que componen la imagen. La segmentación se logra agrupando todos los píxeles con mayor intensidad al umbral en una clase, y los restantes en otra. 
Los algoritmos por umbralado son simples pero efectivos para segmentar imágenes en las que las estructuras de interés cuentan con intensidades, o alguna otra característica cuantificable bien contrastada [Pham et al., 2000]. Normalmente el umbral es introducido por el usuario, pero existen también algunos métodos que permiten obtener el umbral de manera automatica [Sahoo et al.,1988]. 
Su principal limitación radica en que la imagen sólo se divide en dos clases, y en que no es aplicable a imágenes multiespectrales [Pham et al., 2000]. Por otro lado, los algoritmos de umbralado no tienen en cuenta características espaciales de la imagen, lo que hace que sean más sensibles al ruido y a las inhomogeneidades que puedan presentar las intensidades. Por esta razón se han propuesto variaciones al algoritmo clásico de umbralado para la segmentación de imágenes médicas, como en [Lee et al., 1998] que incorporan información basada en intensidades locales y conectividad. En () que proponen un umbralado adaptativo local guiado por conocimiento, para detectar vasos en imagenes de retina.

\subsection{Crecimiento de regiones}
Los algoritmos de segmentación por crecimiento de regiones permiten extraer una región conectada de una imagen basándose en algún criterio de homogeneidad predefinido [Haralick and Shapiro, 1985]. 
En su versión más simple, esta técnica se inicializa mediante la selección de uno o más puntos semilla para cada región que se desea obtener. Luego se incorporan a la región todos los píxeles conectados a esta semilla que poseen características similares. El criterio de crecimiento es una función basada en las similitudes de las características de los pixeles o voxeles, y las características utilizadas para calcular estas similitudes dependen del objetivo del estudio.
Al igual que los métodos de umbralado, los enfoques basados en crecimiento de regiones no suelen utilizarse por sí solos, sino integrados con otros enfoques de segmentación [freixenet2002yet]. Su principal desventaja es que por lo general se requiere la interacción manual del usuario para ubicar las semillas aunque se han desarrollado algunos enfoques para asignar las semillas de manera automática, como en (Kumar y Mehta, 2011) para tumores cerebrales, (Wu et al 2008) para ubicar semillas automáticamente sobre órganos abdominales, y (Pham et al) sobre hígado.

\subsection{Segmentación basada en atlas}
Un atlas está definido como la combinación de una plantilla que es generada utilizando información compilada de la anatomía a segmentar y una imagen segmentada. Luego de registrar la plantilla, esta es utilizada para segmentar la nueva imagen [Cabezas et al., 2011]. Cuando se compara a otros métodos de segmentación, los basados en atlas tienen la capacidad de segmentar imágenes con una relación no bien definida entre regiones e intensidades de píxeles. Otra ventaja importante de este tipo de segmentación es su uso en la práctica clínica para diagnósticos asistidos donde se utilizan generalmente para medir las formas de un objeto o detectar diferencias morfológicas entre un grupo de pacientes [Kalinic, 2009]. Por otro lado, la principal desventaja es el generalmente elevado tiempo necesario para la construcción del atlas. 
La segmentación basada en atlas ha sido utilizada en imágenes de resonancias magnéticas de próstatas [Gubern-Merida and Marti, 2009, Klein et al., 2007, Martin et al., 2010] y también en imágenes de cerebros para fines tales como la detección de epilepsia (Nag et al., Sin año) y análisis de la materia blanca [Lawes et al., 2008].

\subsection{Modelos deformables}
Los algoritmos de segmentación por modelos deformables (también conocidos como contorno deformable o Snakes, para imagenes 2D, o superficies deformables, para imagenes 3D) trabajan deformando un contorno o superficie inicial cerrada, bajo la influencia de fuerzas internas y externas. Dichas fuerzas trabajan estirando o contrayendo este contorno o superficie hasta amoldarse al límite real de la region de interés.
Las fuerzas internas se calculan en base a las características propias del contorno o superficie, y son las que permiten que el contorno o superficie se mantenga suave y continua a medida que se deforma.  Las fuerzas externas son derivadas normalmente de la imagen y son las que atraen al contorno o superficie hacia los bordes de la region de interes. La forma final del contorno o superficie, se obtiene minimizando una funcion de energia [Pham et al., 2000]. Una de las principales desventajas, es que es necesario un contorno o superficie inicial cercano al deseado, y de no tener algoritmos que definan automaticamente esta inicialización, es necesaria la interacción del usuario para dibujarla.
En el capítulo XX se presenta en detalle este algoritmo, donde describimos ademas su implementación como último paso en la propuesta presentada.
\subsection{Clasificadores}
\subsubsection{Supervisados}
\paragraph{k-NN (k-vecinos más cercanos)}
\paragraph{ANN (Redes neuronales artificiales)}
\paragraph{SVM (Support Vector Machine)}
\subsubsection{No supervisados}
\paragraph{K-means}
Este es un método de agrupamiento que busca clasificar n objetos en k grupos, donde k es una cantidad conocida a priori. Cada grupo debe tener un centroide, que inicialmente puede definirse manualmente o automáticamente. Luego, de manera iterativa, se determina la membresía de cada objeto a un cluster, asignándole al grupo que tenga el centroide más cercano y luego de asignados todos los objetos a un cluster, se recalculan los centroides, calculando la media de los objetos pertenecientes a cada grupo. La implementación más común utiliza una técnica de refinamiento iterativo donde se intenta reducir una función de costo y se utiliza la distancia euclídea entre los puntos para definir la pertenencia a un grupo. (macqueen1967some)

\paragraph{k-NN (k-vecinos más cercanos)}
La clasificación de k-vecinos más cercanos es un método basado en reconocimiento de patrones. Este método de segmentación consiste en asignar los puntos de la imagen a una clase, buscando ejemplos en los datos de entrenamiento, con valores similares dentro de un espacio de características predefinido. En este espacio, cada eje representa una de las características del voxel, como por ejemplo ubicación espacial e intensidad del voxel. El conjunto de datos de entrenamiento consiste en muestras preclasificadas que son añadidas al espacio de características de acuerdo a los valores de las mismas. Cada voxel de una imagen nueva, es clasificado comparándolo con las K muestras de entrenamiento más cercanas a él, en términos de distancia euclídea, dentro del espacio de características. La clase más frecuente entre las K muestras de entrenamiento, es la clase que se asigna al voxel [Anbeek et al., 2008].

\paragraph{Fuzzy C-Means}
Fuzzy C-Means es un algoritmo de clustering no supervisado que permite obtener segmentaciones difusas agrupando elementos similares en clusters. Un cluster es un conjunto de elementos que son afines entre sí. Una de las principales desventajas de los algoritmos de clustering tradicionales radica en que los mismos asumen que cada elemento pertenece inequívocamente a un cluster, ignorando si existe o no alguna similaridad con los demás miembros de otros clusters (full1982fuzzy). Una manera de modelar esta similaridad fue introducida por Zadeh en 1965 (Zadeh 1965), y consiste en representar la semejanza de los puntos que se desean clusterizar con una función cuyos valores están entre cero y uno. Basado en esta propuesta, y a diferencia de K-Means, en donde cada elemento pertenece o no a un cluster de manera determinante, en Fuzzy C-Means cada elemento posee una probabilidad de pertenencia a cada uno de los clusters.
En el siguiente capítulo se presenta una explicación detallada de Fuzzy C-Means y ademas se lo utiliza como primer paso en la propuesta de este trabajo.
