\chapter{Conclusiones y trabajos futuros}
En el presente trabajo final de carrera se propuso un método para la segmentación de imágenes médicas tridimensionales basado en la integración de un algoritmo de clustering difuso (Fuzzy C-Means) con un esquema de Modelos Deformables. La combinación de ambas estrategias en un único pipeline de procesamiento permite principalmente la incorporación de información provista por múltiples descriptores al esquema de fuerzas de las Snakes. Para ello, se utiliza como entrada del modelo deformable el mapa de pertenencias correspondiente al tejido a segmentar, obtenido previamente haciendo uso del método de clustering. Esta integración permite incorporar multiplicidad de indicadores sin que sea necesario realizar mayores modificaciones al modelo original utilizado en la literatura. Por otro lado, esto permite hacer uso de una de las mayores ventajas del método de Snakes, que radica en la naturaleza conexa de la membrana deformable. Así, el método completo da lugar a una contribución al área de segmentación no supervisada que complementa ambas estrategias y realiza un aprovechamiento mayor de las características disponibles en la imagen.

Inicialmente se estudió el comportamiento del método de Fuzzy C-Means en diferentes condiciones de ruido y ante la presencia de artefactos característicos de las resonancias magnéticas tridimensionales. Durante estos estudios se demostró empíricamente que el método es capaz de diferenciar intensidades de manera correcta siempre y cuando las mismas no se vean afectadas de manera destacable por el ruido o presenten variabilidades tipo bias no muy pronunciadas. Se demostró, además, que la incorporación de las coordenadas de la imagen como descriptores, lejos de permitir mejorar estas dificultades, empeora notoriamente los resultados, concluyendo así que sería necesario realizar una modificación del modelo completo si se desea tener en cuenta esta información en esta etapa.

Posteriormente se propuso una estrategia de umbralado a partir de los mapas probabilísticos resultantes del algoritmo de Fuzzy C-Means, robusta a la presencia de ruido y que permite obtener mallas sin orificios. La misma consiste en asignarle a cada cluster la etiqueta correspondiente a aquel mapa en donde la probabilidad de pertenencia es mayor, y en un postprocesamiento adicional del volumen binario resultante para remover huecos.

En una tercera etapa, se desarrolló un algoritmo basado en Snakes que permite la deformación de las mallas obtenidas por Fuzzy C-Means, pero utilizando el mapa probabilístico correspondiente a la región de interés en lugar de las intensidades de la imagen. Se incorporó, además, un estudio de sensibilidad de los parámetros del modelo, con el objeto de analizar cómo la variación de cada uno de ellos afecta a los resultados finales.

El esquema completo de segmentación fue evaluado para la segmentación de materia blanca, materia gris y líquido cerebroespinal o cefalorraquideo a partir de un conjunto de imágenes MRI T1-w de cerebros sanos ampliamente utilizada en la literatura. Los valores de coeficiente Dice obtenidos fueron, en promedio, de 0.22 para el líquido cefalorraquideo, 0.70 para la materia gris y 0.75 para la materia blanca. 

Según pudo concluirse a partir del análisis cualitativo y cuantitativo de los resultados, el método es capaz de obtener resultados más que satisfactorios para la materia gris y la materia blanca, aunque los valores de calidad para el líquido cefalorraquídeo requieren ser todavía mejorados. Una posible causa de esta dificultad puede radicar en los propios datos utilizados para la evaluación, que, según se ha indicado en literatura reciente \citep{valverde2015comparison}, presenta algunos errores en el etiquetado. 

El método fue evaluado utilizando las intensidades de la imagen como descriptores de los píxeles, y las intensidades combinadas con los valores de gris obtenidos tras la convolución de la imagen con filtros gaussianos con diferente valor de sigma. Según pudo observarse, la incorporación de este indicador no mejora los resultados obtenidos respecto a la versión basada exclusivamente en intensidades. No es posible, sin embargo, concluir que esto sucede para todo tipo de indicadores, debido sobre todo a que los resultados combinando intensidades y gaussiano mejoran paulatinamente conforme se reduce el valor de sigma, algo que puede estar vinculado a la naturaleza difusa que presentan los bordes luego de aplicado el filtro. En un futuro se prevé la evaluación del pipeline completo pero haciendo uso de descriptores más robustos, tales como información de atlas, descriptores de textura, filtros de realce o indicadores creados ad-hoc para la exclusiva segmentación de tejidos cerebrales.

En general, se demostró también que el método de Snakes permite mejorar ampliamente los resultados obtenidos utilizando solamente Fuzzy C-Means para la segmentación de materia gris. En el caso de la materia blanca se observó una pequeña disminución, que se intuye está relacionada también con la poca capacidad para discriminar algunas regiones del cerebro utilizando únicamente intensidades. La integración de información relativa a la morfología de los tejidos permitiría, en un futuro, mejorar esta performance notoriamente.

Entre los trabajos futuros que pueden desprenderse de esta tesis de grado, uno de los más inmediatos debería basarse en la utilización de un algoritmo inicial más robusto que Fuzzy C-Means. En particular, la incorporación de los scores obtenidos mediante el uso de un clasificador supervisado (por ejemplo, Support Vector Machines) en reemplazo de los mapas probabilísticos permitiría mejorar los resultados en situaciones en las que los descriptores no llegan a ser tan certeros como es de esperar. Por otro lado, un análisis general y cualitativo del costo computacional del prototipo Matlab desarrollado para este trabajo sugiere el desarrollo futuro de mejoras basadas en cálculo paralelo, ya sea mediante el uso de múltiples threads a correr en los diferentes núcleos de la CPU, o bien la migración y paralelización de la implementación en GPUs.

Con posterioridad a la presentación de este trabajo se prevé, además, la puesta a disponibilidad del código fuente de este prototipo, a los efectos de contribuir con la comunidad de desarrolladores Matlab a través del sitio File Exchange de Mathworks.